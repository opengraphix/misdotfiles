%
%    Escribe aqu� tu nombre completo
%
%
\documentclass{article}
%%%%%%%%%%%%%%%%%%%%%%%%%%%%%%%%%%%%%%%%%%%%%%%%%%%%%%%%%%%%%%%%%%%%%%%%%%%%%


\begin{document}


\begin{center}
{\Large Problemas de C\'{a}lculo}
\medskip

{\Large Autor: Ren\'{e} D\'{a}ger}
\end{center}

\bigskip
%
%
%%%%%%%%%%%%%%%%%%%%%%%%%% Problema 1 %%%%%%%%%%%%%%%%%%%%%%%
%
%
\noindent\fbox{\begin{minipage}{5in} \textbf{Problema 1}
%
Sea $f:(1,+\infty )\rightarrow (0,1)$ la funci\'{o}n definida por
$$
f(x)=\frac{x-1}{x}.
$$

a) Demostrar que $f$ es biyectiva.

b) Calcular la funci\'{o}n inversa de $f$.
%
\end{minipage}}

\bigskip
 \textbf{Soluci\'{o}n}  a) Para comprobar que $f$ es
biyectiva debemos probar que es inyectiva y sobreyectiva en los
intervalos en que est\'{a} definida.

\emph{Inyectividad:} Sean $x,y$ dos puntos tales que $f(x)=f(y)$.
Esto
significa%
$$
\frac{x-1}{x}=\frac{y-1}{y},
$$%
de donde%
$$
y(x-1)=x(y-1),
$$%
$$
yx-y=xy-x,
$$%
es decir $x=y$. As\'{\i} podemos concluir que $f$ es inyectiva.

\emph{Sobreyectividad}: Debemos comprobar que para cada $y_{0}\in
(0,1)$
existe $x_{0}\in (1,+\infty )$ tal que $f(x_{0})=y_{0}$. Esto es,%
$$
\frac{x_{0}-1}{x_{0}}=y_{0}.
$$%
Despejando resulta%
$$
x_{0}=\frac{1}{1-y_{0}}.
$$%
As\'{\i}, hemos obtenido un valor $x_{0}$ tal que
$f(x_{0})=y_{0}$. Ahora basta notar que si $y_{0}\in (0,1)$
entonces $x_{0}>1$.

b) Del c\'{a}lculo del punto a) concluimos que
$$
f^{-1}(x)=\frac{1}{1-x}.
$$
\bigskip

%
%
%%%%%%%%%%%%%%%%%%%%%%%%%% Problema 2 %%%%%%%%%%%%%%%%%%%%%%%
%
%
\noindent\fbox{\begin{minipage}{5in} \textbf{Problema 2}
%
%   escribir aqu� el texto del problema 2
%
%
\end{minipage}}

\bigskip
 \textbf{Soluci\'{o}n}

%
%    escribir aqu� la soluci�n del problema 2
%
%
\bigskip

%
%
%%%%%%%%%%%%%%%%%%%%%%%%%% Problema 3 %%%%%%%%%%%%%%%%%%%%%%%
%
%
\noindent\fbox{\begin{minipage}{5in} \textbf{Problema 3}
%
Demostrar utilizando la definici\'{o}n de l\'{\i}mite que
$$
\lim_{n\rightarrow \infty }\frac{n+1}{n}=1.
$$
%
\end{minipage}}

\bigskip
 \textbf{Soluci\'{o}n}
%
Debemos probar que para todo $\varepsilon >0$ existe
un n\'{u}mero $N(\varepsilon )$ tal que si $n>N(\varepsilon )$ entonces%
$$
\left\vert \frac{n+1}{n}-1\right\vert <\varepsilon .
$$
Pero esta \'{u}ltima desigualdad es equivalente a
$$
-\varepsilon <\frac{n+1}{n}-1<\varepsilon ,
$$
es decir,%
$$
-\varepsilon <\frac{1}{n}<\varepsilon .
$$
De aqu\'{\i} resulta%
$$
n>\frac{1}{\varepsilon }
$$
por eso basta escoger $N(\varepsilon )=\frac{1}{\varepsilon }$.

En efecto, si  $n>N(\varepsilon )$ entonces
$$
\left\vert \frac{n+1}{n}-1\right\vert <\left\vert \frac{1}{n}\right\vert <%
\frac{1}{N(\varepsilon )}=\frac{1}{\frac{1}{\varepsilon
}}=\varepsilon .
$$

%

\bigskip

%
%
%%%%%%%%%%%%%%%%%%%%%%%%%% Problema 4 %%%%%%%%%%%%%%%%%%%%%%%
%
%
\noindent\fbox{\begin{minipage}{5in} \textbf{Problema 4}
%
%   escribir aqu� el texto del problema 4
%
%
\end{minipage}}

\bigskip
 \textbf{Soluci\'{o}n}

%
%    escribir aqu� la soluci�n del problema 4
%
%
\bigskip

%
%
%%%%%%%%%%%%%%%%%%%%%%%%%% Problema 5 %%%%%%%%%%%%%%%%%%%%%%%
%
%
\noindent\fbox{\begin{minipage}{5in} \textbf{Problema 5}
%
Sea $y(x)$ la funci\'{o}n definida impl\'{\i}citamente por la
igualdad%
$$
x^{2}y^{2}-\cos (x^{2}+y^{2})=0.
$$
Calcular $y^{\prime }(0)$.
%
\end{minipage}}

\bigskip
 \textbf{Soluci\'{o}n}

%
Sustituyendo $x=0$ en la
igualdad que define la funci\'{o}n obtenemos%
$$
0^{2}\left( y(0)\right) ^{2}-\cos (0^{2}+\left( y\left( 0\right)
\right) ^{2})=0,
$$%
de donde resulta%
$$
\cos \left( y\left( 0\right) \right) ^{2}=0.
$$%
De aqu\'{\i}, en particular obtenemos que $y(0)\neq 0$, \  $\sin
y(0)\neq 0$.

Si derivamos la igualdad con respecto a $x$, se obtiene%
$$
2xy^{2}+2x^{2}yy^{\prime }-\left( 2x+2yy^{\prime }\right)
\sin\left( x^{2}+y^{2}\right) =0.
$$%
Sustituyendo $x=0$ resulta%
$$
2y(0)y^{\prime} (0)\sin\left( y(0)\right) ^{2}=0,
$$%
por lo que, en vistas de que $y(0)\neq 0$, \  $\sin\left(
y(0)\right) ^{2}\neq 0$,
$$
y^{\prime} (0)=0.
$$

%

\bigskip

%
%
%%%%%%%%%%%%%%%%%%%%%%%%%% Problema 6 %%%%%%%%%%%%%%%%%%%%%%%
%
%
\noindent\fbox{\begin{minipage}{5in} \textbf{Problema 6}
%
%   escribir aqu� el texto del problema 6
%
%
\end{minipage}}

\bigskip
 \textbf{Soluci\'{o}n}

%
%    escribir aqu� la soluci�n del problema 6
%
%
\bigskip

%
%
%%%%%%%%%%%%%%%%%%%%%%%%%% Problema 7 %%%%%%%%%%%%%%%%%%%%%%%
%
%
\noindent\fbox{\begin{minipage}{5in} \textbf{Problema 7}
%
Sea $f(x)=3x-2$ y denotemos por $s_{n}(f)$ la suma de Riemann
correspondiente a los extremos derechos en una partici\'{o}n
uniforme del intervalo $(-2,2)$ en $n$ subintervalos.

a) Calcular $s_{4}(f).$

b) Determinar el error que se comete al aproximar
$$
\int_{-2}^{2}f(x)dx
$$%
por $s_{4}(f).$
%
\end{minipage}}

\bigskip
 \textbf{Soluci\'{o}n}

%
a) Tenemos que $a=-2,\quad
b=2,$ $n=4,$ por lo que%
\begin{eqnarray*}
h &=&\frac{b-a}{n}=\frac{4}{4}=1, \\
x_{0} &=&-2,\qquad x_{1}=-1,\qquad x_{2}=0,\qquad x_{3}=1,\qquad
x_{4}=2.
\end{eqnarray*}%
Como escogemos los extremos derechos tendremos
$$
\xi _{0}=-1,\qquad \xi _{1}=0,\qquad \xi _{2}=1,\qquad \xi _{3}=2,
$$%
$$
f\left( \xi _{0}\right) =-5,\qquad f\left( \xi _{1}\right)
=-2,\qquad f\left( \xi _{2}\right) =1,\qquad f\left( \xi
_{3}\right) =4.
$$

Entonces%
$$
s_{4}(f)=-5-2+1+4=-2.
$$

b) Como
$$
\int_{-2}^{2}f(x)dx=\int_{-2}^{2}(3x-2)dx=\left. \left( \frac{3}{2}%
x^{2}-2x\right) \right\vert _{-2}^{2}=-8
$$%
resulta%
$$
\left\vert \int_{-1}^{1}f(x)dx-s_{4}(f)\right\vert =\left\vert
-8+2\right\vert =6.
$$%
El error que se comete es igual a $6.$

%
\bigskip

%
%
%%%%%%%%%%%%%%%%%%%%%%%%%% Problema 8 %%%%%%%%%%%%%%%%%%%%%%%
%
%
\noindent\fbox{\begin{minipage}{5in} \textbf{Problema 8}
%
%   escribir aqu� el texto del problema 8
%
%
\end{minipage}}

\bigskip
 \textbf{Soluci\'{o}n}

%
%    escribir aqu� la soluci�n del problema 8
%
%
\bigskip
%
%
%%%%%%%%%%%%%%%%%%%%%%%%%%%%%%%%%%%%%%%%%%%%%%%%%%%%%%%%%%%%
%
%
\end{document}
