%\documentclass[options]{class}
\documentclass[10pt,journal]{IEEEtran}

%Paquete de Idioma
\usepackage[spanish]{babel}

%Codificación Alfabeto
\usepackage[utf8]{inputenc}

%Codificación de Fuente
\usepackage[T1]{fontenc}

%Índice
\usepackage{makeidx}

%Gráficos
\usepackage{graphicx}
\usepackage{float} 
%\usepackage{xcolor} 

%Matemática
\usepackage{amsmath}
\usepackage{amsfonts}
\usepackage{amssymb}
%\usepackage{amstext} 

%Estilo de Página Numeración superior
%\pagestyle{headings}

%Hiperlinks \href{url}{text}
\usepackage[pdftex]{hyperref}


\begin{document}

%Titulo
\title{Título del Documento}

%Autor
\author{Universidad de San Carlos, Facultad de Ingeniería\\
Departamento de Física\\ Laboratorio de Física Básica, Física 1, Física 2, Física 3 o Física 4\\
0000-00000 Nombre del estudiante1\\ 0000-00000 Nombre del estudiante2\\ 0000-00000 Nombre del estudiante3 }




\maketitle{}  

%Resumen
\begin{abstract}
  
Un buen resumen debe permitir al lector identificar, en forma rápida y precisa, el
contenido básico del trabajo; no debe tener más de 250 palabras y debe redactarse en
pasado, exceptuando el último párrafo o frase concluyente. No debe aportar información
o conclusión que no está presente en el texto, así como tampoco debe citar referencias
bibliográficas. Debe quedar claro el problema que se investiga y el objetivo del mismo.

\end{abstract}

\section{Objetivos}
 
Es necesario indicar de manera el propósito del trabajo. Definir los objetivos de la
práctica permite la formulación de una o varias hipótesis. Los objetivos se pueden
clasificar en objetivos generales y específicos.


\section{Marco Teórico}
Su contenido debe tener una exposición lógica y ordenada de los temas, así como
evitar la excesiva extensión y el resumen extremo de la presentación de la teoría. Es
importante que la teoría expuesta no sea una “transcripción bibliográfica” de temas
que tengan alguna relación con el problema, sino que fundamente científicamente el
trabajo.

\section{Diseño Experimental}

Hace una descripción del método o técnica utilizada para medir y/o calcular las
magnitudes físicas en estudio, y si es del caso, del aparato de medición. Hay que
recordar que el “método” es el procedimiento o dirección que conducirá a la solución
del problema planteado. Se recomienda redactar una breve introducción para explicar
el enfoque metodológico seleccionado.

\section{Resultados}

Los resultados se analizan, en general, por medio de gráficos o diagramas,
debidamente identificados, que muestran el comportamiento entre las magnitudes
medidas o que permiten calcular otras magnitudes. Dependiendo de lo extenso de las
gráficas y/o tablas, éstas se pueden anexar al final del trabajo.

Todos los datos obtenidos deben ir acompañados de las unidades dimensionales,
con su debida incertidumbre de medida, que mostrarán la calidad, precisión y
reproductibilidad de las mediciones. Éstos deben ser consistentes, a lo largo del
reporte.

\section{Discusión de Resultados}

En este apartado se deben analizar los resultados obtenidos, contrastándolos con
la teoría expuesta en la sección del Marco Teórico. Corresponde explicar el
comportamiento de las tablas y gráficas expuestas en la sección de Resultados,
tomando en cuenta el análisis estadístico apropiado.

\section{Conclusiones}

Las conclusiones son interpretaciones lógicas del análisis de resultados, que
deben ser consistentes con los objetivos presentados previamente.

\section{Fuentes de consulta}

Las fuentes de consulta se citan en forma organizada y homogénea, tanto de los
libros, de los artículos y, en general, de las obras consultadas, que fueron
indispensables indicar o referir en el contenido del trabajo

\section{Anexos}

Se incluyen gráficas, ilustraciones, cálculos, etc.

\end{document}


%Código necesario para insertar una imagen:
%\begin{figure}[H]
%\centering
%\includegraphics[scale=0.7]{nombre de la imagen}
%\end{figure}

% el simbolo de porcentaje no se debe de incluir este solo sirve para dejar mensajes.

% donde dice scale el valor varia de 0 a 1 donde 1 es el tamaño real de la imagen y 0 equivale a un punto

%donde dice nombre de la imagen colo car el nombre de la imagen junto con su extension ejemplo  "imagen1.png"






