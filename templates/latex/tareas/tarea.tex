\documentclass[12pt]{article}%
\usepackage{amsfonts}
\usepackage{fancyhdr}
\usepackage{comment}
\usepackage[a4paper, top=2.5cm, bottom=2.5cm, left=2.2cm, right=2.2cm]%
{geometry}
\usepackage{times}
\usepackage{amsmath}
\usepackage{changepage}
\usepackage{amssymb}
\usepackage{graphicx}%
\setcounter{MaxMatrixCols}{30}
\newtheorem{theorem}{Theorem}
\newtheorem{acknowledgement}[theorem]{Acknowledgement}
\newtheorem{algorithm}[theorem]{Algorithm}
\newtheorem{axiom}{Axiom}
\newtheorem{case}[theorem]{Case}
\newtheorem{claim}[theorem]{Claim}
\newtheorem{conclusion}[theorem]{Conclusion}
\newtheorem{condition}[theorem]{Condition}
\newtheorem{conjecture}[theorem]{Conjecture}
\newtheorem{corollary}[theorem]{Corollary}
\newtheorem{criterion}[theorem]{Criterion}
\newtheorem{definition}[theorem]{Definition}
\newtheorem{example}[theorem]{Example}
\newtheorem{exercise}[theorem]{Exercise}
\newtheorem{lemma}[theorem]{Lemma}
\newtheorem{notation}[theorem]{Notation}
\newtheorem{problem}[theorem]{Problem}
\newtheorem{proposition}[theorem]{Proposition}
\newtheorem{remark}[theorem]{Remark}
\newtheorem{solution}[theorem]{Solution}
\newtheorem{summary}[theorem]{Summary}
\newenvironment{proof}[1][Proof]{\textbf{#1.} }{\ \rule{0.5em}{0.5em}}

\newcommand{\Q}{\mathbb{Q}}
\newcommand{\R}{\mathbb{R}}
\newcommand{\C}{\mathbb{C}}
\newcommand{\Z}{\mathbb{Z}}

\begin{document}

\title{Homework}
\author{Jaime Carlos M Infante}
\date{\today}
\maketitle
\section{Example} 


        

All problems like the following lead eventually to an equation in that simple form.
Cada una de las cosas.
Como se dice.
 

\subsection{Problem 1}
Jane spent \$42 for shoes. This was \$14 less than twice what she spent for a blouse. How much was the blouse?
\subsection{Solution}
Every word problem has an "unknown number".In this problem,it is the price of the blouse. Always let "x" represent the "unknown number".That is, let "x" answer the question.
\subsection{Solution part 2}
Let x,then,be how much she spent for the blouse.The problem states that "This"--that is, \$42--was \$14 less than two times x.

           Here is the Equation: 2x-14= 42
                                 
                                 2x=42+14
                                 
                                 =56
                                 
                                 x=56/2
                                 
                                 =28





\section{Example 2}
There are "b" boys in the class. This is three more than the number four times the number of girls.
\subsection {Solution}
Again let "x" represent the unknown number that you are asked to find: Let X be the number of girls. The problem states that "This"--b--is three more than 4 times X. 
          
          4x+3=b
          
          4x=b-3
          
          x=b-3/4
          
 The solution here is not a number,because it will depend on the value of b. This is a type of a literal equation, which is very common in algaebra.





\end{document}